%%%%%
%Metal Room
%%%%%

%%%%%%%%%%%%%%%%%%%%%%%%%%%%%%%%%%%%%%%%%%
%: Metal shop Pillar Drill
\machinePage{Metal Pillar Drill} %title of machine.
	{NoMandatory, Two}%safety options: Induction, blank (instruction mandatory), NoMandatory
		% Additionally, options include: Online, InPerson, Log, Dust, Two
	{Make sure you clamp your workpiece well - to avoid spinning.

Be aware of other people around you.
} %further text on mandatory safety rules, appears at top of page with Log warning, Dust warning, Induction, etc.
	{%Main warnings, alerts, etc.
%	\action{49}{General mandatory action sign}{} %Exclamation mark.
%	\action{50}{Refer to instruction manual/booklet}{} %Person reading book.
%	\action{51}{Wear ear protection}{} %Person wearing ear defenders.
	\action{52}{Eye protection recommended}{Swarf will fly. Marterial can come loose.} %Person wearing safety glasses.
%	\action{55}{Opaque eye protection must be worn}{} %Person wearing opaque safety glasses.
%	\action{56}{Wear safety footwear}{} %Safety boots.
%	\action{57}{Wear protective gloves}{} %Gloves.
	\action{58}{Wear protective clothing}{Wear tight fitting clothing, keep long hair tied up, no jewellery or anything else that can be caught by the machine.} %symbol of overalls.
%	\action{61}{Wear a face shield}{} %head wearing a face shield.
	\action{64}{Respiratory protection is recommended}{} %head wearing a covid style mask.
%	\action{67}{Wear a welding mask}{} %head wearing a welding mask.
%	\action{74}{Use protective apron}{} %person wearing an apron.
%	\prohib{75}{General prohibition sign}{} %general prohibition sign with nothing in it.
%	\warn{77}{no open flames}{} %lit match prohibition sign.  This is a "Prohib" sign but needed a different offset, so we use the Warn command.
%	\warn{88}{No reaching in}{} %hand between two converging lines. This is a "Prohib" sign but needed a different offset, so we use the Warn command.
	\prohib{100}{Do not wear gloves}{Especially strong leather ones. They get caught easily and then `help' ripping body parts off.} %gloves prohibition sign.
%	\prohib{107}{General warning sign}{} %Exclamation mark in warning triangle. This is an "alert/Warn" sign but needed a different offset, so we use the warn command.
%	\warn{110}{Laser beam}{} %Laser beam warning triangle.
%	\warn{117}{Slippery surface}{} %Human figure falling backwards warning triangle.
%	\warn{123}{Hot Surface}{} %hot surface warning triangle.
%	\warn{124}{Automatic start-up}{} %spinny thing go fast warning triangle.
%	\warn{127}{flammable material}{} %flame warning trianble.
%	\warn{128}{Sharp element}{} %bandaged hand above sharp point.
	\warn{130}{Crushing}{Risk of crushing -- the machine does not have any sensors that detect obstacles. The drill is operated by gears. They will not slip or stop.} %hands getting crushed.
% Other symbols are available in the ISO PDF in the github. Extract the actual pagenumber. 
	}
	{%Final footer warning
	\textbf{Hints and Tips}
  Cutting oil is your friend.
  Drill a pilot hole using smaller bits, then work your way up in size.
  If you run out of bits on the wall, top up from the drill bit drawer. If the drawer runs out, check the consumables cupboard or order more. The hackspace will reimburse you.
  Sheet metal: Don't use a normal twist drill. Use a step drill instead. Ensure the work is well clamped.
  
   https://wiki.bristolhackspace.org/equipment/woodshop/floor-standing-pillar-drill
	
	}


%%%%%%%%%%%%%%%%%%%%%%%%%%%%%%%%%%%%%%%%%%
%: Metal Lathe
\machinePage{Metal Lathe}{Induction, Two}{
Be careful when using the auto-feed; especially with cutters near the revolving head.
}{
\prohib{100}{Do not wear gloves}{Especially strong leather ones. They get caught easily and then `help' ripping body parts off.}
\action{52}{Eye protection recommended}{Swarf will fly. Cutters can come loose. Broken cutters can fly very far and are razor sharp.}
\action{58}{Wear protective clothing}{Wear tight fitting clothing, keep long hair tied up, no jewellery or anything else that can be caught by the machine.}
\warn{125}{Crushing}{Risk of crushing -- the machine does not have any sensors that detect obstacles.

Note that it can also run very slow; barely noticeable.}
\warn{117}{Slippery surface}{Floor may get slippery when using coolant.}
% \warn{128}{Sharp rotating elements}{This machine is essentially a cutter without any guard whatsoever.}
}{
Clean after use -- especially when you have used coolant or have used metals that can rust easily or cause a galvanic reaction.
}

%%%%%%%%%%%%%%%%%%%%%%%%%%%%%%%%%%%%%%%%%%
%: Slijpsteen
\machinePage{Large Grinder, Two}{}{}{
\action{58}{Wear protective clothing}{Wear tight fitting clothing, keep long hair tied up, no jewellery or anything else that can be caught by the machine.}
\action{52}{Eye protection needed}{}
\action{51}{Ear protection recommended}{}
}{}

%%%%%%%%%%%%%%%%%%%%%%%%%%%%%%%%%%%%%%%%%%
%: Metaal zaag
\machinePage{Metal bandsaw, Two}{}{
Use plenty of cutting oil, \textsc{wd40} or similar. 

Do not leave unattended (not in the least as you may want to keep lubricating it to get a nice cut).
}{
\action{58}{Wear protective clothing}{Wear tight fitting clothing, keep long hair tied up, no jewellery or anything else that can be caught by the machine.}
\warn{128}{Sharp rotating elements}{}
\action{52}{Eye protection recommended}{}
}{
}

%%%%%%%%%%%%%%%%%%%%%%%%%%%%%%%%%%%%%%%%%%
%: Metal Bandsaw
\machinePage{Metal bandsaw, Two}{}{
Suitable for metal; hard wood and some types of plastics. 
}{
\action{58}{Wear protective clothing}{Wear tight fitting clothing, keep long hair tied up, no jewellery or anything else that can be caught by the machine.}
\action{52}{Eye protection recommended}{}
\action{51}{Ear protection recommended}{Note also that this machine should not be used after 19:00 if it is that noisy}
\warn{128}{Sharp rotating elements}{Bypassing or using your fingers to hold the guard open is downright stupid.}
}{
Use of cutting oil recommended !
% No special comments/instructions.
}

%%%%%%%%%%%%%%%%%%%%%%%%%%%%%%%%%%%%%%%%%%
%: Hembrug Metal Drill Press
\machinePage{Hembrug Drill press}{Mandatory, Two}{
Make sure you clamp your workpiece well - to avoid spinning.

Let the drill do its work - do not apply much pressure (drill gets dull or will snaps).

Use the correct drill speed (higher for aluminium, low for steel, very low for stainless steel (\textsc{rvs}). Use the table on the wall.

Use drilling oil when needed (when in doubt - always use the green oil).
}{
\action{52}{Eye protection recommended}{Sharp bits will fly, drills snap.}
\action{58}{Wear protective clothing}{Wear tight fitting clothing, keep long hair tied up, no jewellery or anything else that can be caught by the machine.}
\prohib{100}{Wearing gloves is forbidden}{Especially strong leather ones. They get caught easily and then `help' ripping body parts off. 
}}
{
\textbf{Beware that this is a powerful, 3-phase (Krachtstroom), machine.}

So unlike the other pillar-drill - it won't stop of something jams.

Beware that the machine can start to spin if you press the green button while the rotary knob is set to 1 or 2. 

Only use the emergency button on the front for emergency stops. For routine on-off use the rotary knob. De-energise the machine with the red button when you are done.

Noise wise -- be considerate to our neighbours -- especially in the evenings.
}


%%%%%%%%%%%%%%%%%%%%%%%%%%%%%%%%%%%%%%%%%%
%: Drill & Mill
\machinePage{Drill \& Mill}{NoMandatory, Two}{
\textbf{Drilling}: Use with common sense, do not drill in the metal of the XY table! Use the wooden overlay to protect it. 

\textbf{Milling}: Getting instructions \textbf{prior} to use is mandatory. After milling: put the machine back in order for drilling, so others can use it

\textbf{When changing the drillbit for milling: use a nylon hamerhead, do NOT use a metal hamer.  }
}{
\prohib{100}{Do not wear gloves}{Especially strong leather ones. They get caught easily and then `help' ripping body parts off. 

Very thin (latex) gloves that easily tear are ok.}
\action{52}{Eye protection recommended}{Especially when milling. Swarf will fly. Cutters can come loose. Broken cutters can fly very far and are razor sharp.}
\action{58}{Wear protective clothing}{Wear tight fitting clothing, keep long hair tied up, no jewellery or anything else that can be caught by the machine.}
\warn{128}{Sharp rotating elements}{This machine is essentially a cutter without any guard whatsoever.}
\warn{117}{Slippery surface}{Floor will get very slippery; especially when using coolant.}
}{
To adjust the height: 1) loosen the lower nut at the right back side of the machine (marked: 'deze moer' ). Then 2) use the big handle on the left side to crank up/lower the whole top part of the drill. \\
\textbf{And 3) fasten the nut again before drilling!}

For speed adjustment open the top and re-arrange the belts as shown on the diagram on the machine for the right speed. You can run these off/on the wheels by starting and ending with the larger of the two.
}

%%%%%%%%%%%%%%%%%%%%%%%%%%%%%%%%%%%%%%%%%%
%: Metal Sander
\machinePage{Metal Sander}{Two}{
This machine is for metal sanding - both ferrous and non-ferrous metals are allowed (until further notice).

\textbf{Connect a shopvac/vacuum cleaner to keep the dust under control.}

}{
\action{58}{Wear protective clothing}{Wear tight fitting clothing, keep long hair tied up, no jewellery or anything else that can be caught by the machine.}
\action{51}{Ear protection recommended}{Note also that this machine should not be used after 19:00 -- it is that noisy!}
\action{52}{Eye protection recommended}{}
\action{64}{Wear a mask when needed}{And keep the dust under control by connecting the vacuum cleaner or a shop vac. The grit is both unhealthy \& damaging to other equipment.}
\warn{128}{High speed abrasive surfaces}{So keep your fingers away.  

It takes metal away awfully fast.}{}
}
{Consult the WIKI (or ask) when you need to change the sanding belt. Report this on the mailing list.

Try to leave the machine in at least as clean a state as you found it. Likewise, if you `gum up' the sanding belt (easy with for example soft aluminium) -- do feel encouraged to replace it or at least warn the next user via the mailing list.
}


%%%%%%%%%%%%%%%%%%%%%%%%%%%%%%%%%%%%%%%%%%
%:   Drill press
\machinePage{Oscillating Band/Bobbin Sander}{NoMandatory, Two}{
This machine is for WOOD only.

Make sure that the bobbin/band is fixed in position and cannot slip. 

Let the sander do its work - do not apply much pressure (things will burn, you'll ruin the mechanics).
}{
\action{49}{Dust extraction mandatory}{Use the dust extractor. Always.}
\action{64}{Wear a mask when needed}{Depending on the wood being sanded.}
\action{52}{Eye protection recommended}{}
\action{58}{Wear protective clothing}{Wear tight fitting clothing, keep long hair tied up, no jewellery or anything else that can be caught by the machine.}
}{
Noise wise -- be considerate to our neighbours -- especially in the evenings.
}

